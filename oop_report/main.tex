\documentclass[dvipdfmx]{jsarticle} % 文書クラスの設定
\usepackage[T1]{fontenc} % フォントの設定
\usepackage{lmodern} % フォントの設定
% この記事〈https://qiita.com/zr_tex8r/items/297154ca924749e62471〉に,フォント設定の詳細な説明がまとまっていました.
\usepackage{multicol} % 多段組
\usepackage{amsthm,amsmath,amssymb,amsfonts} % ams系
\usepackage{latexsym} % 数式で使える記号を増やす
\usepackage{mathrsfs} % 花文字など
\usepackage{mathtools} % 数学系のいろいろ
\usepackage[dvipdfmx]{graphicx,xcolor} % グラフィックと色
\usepackage{float,wrapfig} % 図表の配置
\usepackage{booktabs,multirow} % 表
\usepackage{appendix} % 付録
\usepackage{listings,jlisting} % ソースコード
\usepackage[dvipdfmx]{hyperref} % ハイパーリンク

\hypersetup{%
 setpagesize=false,%
 bookmarks=true,%
 bookmarksdepth=tocdepth,%
 bookmarksnumbered=true,%
 colorlinks=false,%
 pdftitle={},%
 pdfsubject={},%
 pdfauthor={},%
 pdfkeywords={}}

\theoremstyle{definition} % 定理環境のスタイル設定
\newtheorem{theorem}{定理}[section]
\newtheorem{lemma}[theorem]{補題}
\newtheorem{corollary}[theorem]{系}
\newtheorem{definition}[theorem]{定義}
\newtheorem{remark}[theorem]{注}
\newcommand{\x}{$\mathbb{X}$} % マクロの作成(新しいコマンドの定義)

% タイトルの設定
\title{タイトル}
\author{$5423041$ 長谷川絢南 \and $5423052$ 渡邊陽翔 \and $5423072$ 境田悠希}
% \author{太郎\thanks{○大学} \and 次郎\thanks{●大学院} \and 三郎\thanks{株式会社△}} % 著者が複数,注意書きが必要の場合
\date{\today}

\begin{document} % 本文の開始

\maketitle % タイトルを出力

\begin{abstract} % 概要
  
\end{abstract}

\section{システムの概要(外部仕様)}

\section{クラス設計の指針}

\section{クラス設計の詳細}

\section{実行結果}

\section{クラス設計に対する考察}

\section{}

\appendix % 以降の節が付録になる

\section{メンバーの誰が何を担当したか}

\section{ソースコード}

% ソースコードの表示設定
\lstset{
  frame=single, % 枠で囲む
  numbers=left, % 左に行番号を表示
  breaklines = true % 長い行の改行
}

% ソースコードの挿入
\begin{lstlisting}[language=Python,caption=hello\_world.py,label=code:hello_world]
message = "Hello World!"
print(message)
\end{lstlisting}

\end{document}
